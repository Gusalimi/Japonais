\documentclass[a4paper]{report}
\usepackage[utf8]{inputenc}
\usepackage[T1]{fontenc}
\usepackage[french]{babel}
\usepackage{CJKutf8}
\usepackage{textcomp}
\usepackage{caption}
\usepackage{blindtext}
\usepackage{xcolor}

\usepackage{hyperref}
\hypersetup{
    colorlinks=true,
    linkcolor=black,
    urlcolor=cyan,
    pdftitle={Cours de Japonais},
}

\usepackage{helvet}
\renewcommand{\familydefault}{\sfdefault}

\usepackage{fancyhdr}
\pagestyle{fancy}
\fancyhead[L]{}
\fancyhead[C]{\leftmark}
\fancyhead[R]{}
\setlength{\headheight}{13pt}

\title{Cours de Japonais}
\author{Guillaume \textsc{Sailé} \thanks{Basé sur les \href{https://www.youtube.com/channel/UChFfLNTK64xQj7NscGmLLLg}{cours de Julien Fontanier}}}
\date{}

\begin{document}
\begin{CJK}{UTF8}{goth}
\maketitle
\tableofcontents
\part{Les bases du japonais}
\chapter{Les systèmes d'écritures japonais}
Il existe plusieurs alphabets en japonais, les kana (かな) et les kanji (漢字).

Voyons d'abord les かな.
\section{Les かな}
\subsection{Les ひらがな}
\begin{table}[h]
\centering
\begin{tabular}{c|c c c c c}
    & -a & - i & - u & - e & - o\\
    \hline
    & あ & い & う & え & お\\
 k- & か & き & く & け & こ\\
 s- & さ & し & す & せ & そ\\
\end{tabular}
\caption*{Tableau des Hiragana}
\end{table}

\subsection{Les カタカナ}
\begin{table}[h]
\centering
\begin{tabular}{c|c c c c c}
    & -a & -i & -u & -e & -o\\
    \hline
    & ア & イ & ウ & エ & オ\\
 k- & カ & キ & ク & ケ & コ\\
 s- & サ & シ & ス & セ & ソ\\
\end{tabular}
\caption*{Tableau des Katakana}
\end{table}

\blindtext

\blindtext

\end{CJK}
\end{document}
